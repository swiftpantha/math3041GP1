\documentclass[]{article}
\usepackage{amsmath,amsfonts,amssymb,fancyhdr, enumerate, graphicx}
\usepackage[bottom]{footmisc}
\usepackage{setspace}
\usepackage{cite}
\doublespacing

\pagestyle{fancy}  
\oddsidemargin 1cm
\hoffset-1cm
\voffset-0.5cm
\topmargin-1.4cm
\textheight 24cm \textwidth 16.5cm \parindent 0.5cm

\begin{document}

\title{Math 2301 - Summaries}
\author{Mossy}
\date{Today}
\maketitle

\section*{Floating Point Aritmetic}
\subsection*{Representation}
Every real number $x$ has a floating point representation.
\[x = S b^e\]
\begin{itemize}
\item $S = significand$
\item $b = base$
\item $e = exponent$
\end{itemize}


\subsection*{Binary}
\subsubsection*{Integer Conversion}
\[
\begin{array}{ccccccc}
x & 50 & 25 & 12 & 6 & 3 & 1\\
\mod(x,2) & 0 & 1 & 0 & 0 & 1 & 1
\end{array}
\]

\subsubsection*{Non Integer Conversion}
\[
\begin{array}{c c c c c c}
x & 0.3125 & 0.625 & 0.25 & 0.5 & 0\\
2x > 1 & 0 & 0 & 1 & 0 & 1
\end{array}
\]
\pagebreak

\subsubsection*{Rounding Error Analysis}
$\varepsilon$ is the interval in the computer. A number is rounded to the nearest number able to be composed of an integer number of $\varepsilon$. You either round up or down unless it is exactly in between the two. In this case if $d_{p-1}$ is odd you round up if even down.

\begin{itemize}
\item error in $x_*=x_* - x$
\item absolute error in $x_* = |x_*-x|$
\item relative error in $x_* = \frac{|x_*-x|}{|x|}$
\end{itemize}

\subsubsection*{Signifcant Digits}
We define the number of significant figures to be
\[max_{d\in \mathbb Z }\left(\frac{|x_*-x|}{|x|}<0.5\times 10^{-d}\right) \]




\end{document}